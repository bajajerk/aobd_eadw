{\color{gray} For each of the following queries, identify one possible reason why an optimizer might not find a good
execution plan. Rewrite the query so that a good plan is likely to be found. Any available indexes, or known
constraints, are listed before each query.}

{\color{gray}\subsection{} An index is available on the calories attribute:}

\begin{lstlisting}
SELECT type
FROM CHEESE
WHERE calories * 100 < 800
\end{lstlisting}

The optimizer might not find a good execution plan because the index over calories times 100 might not exist, a good solution is to divide 800 by 100.

\begin{lstlisting}
SELECT type
FROM CHEESE
WHERE calories < 8
\end{lstlisting}

It is important that the index is a b+ tree so the search for cheeses with calories lower then 8 is sequential after the first tree traversal.

{\color{gray}\subsection{} An index is available on the calories attribute:}

\begin{lstlisting}
SELECT type
FROM CHEESE
WHERE calories<90 AND calories>40
\end{lstlisting}

{\color{red}XXXXXXXXXXXXXXXXXXXXXXXXXXXXXXXXXXXXXXXXX}\footnote{BETWEEN}

{\color{gray}\subsection{} A B+ Tree index is available on the calories attribute:}

\begin{lstlisting}
SELECT type
FROM CHEESE
WHERE calories=90 OR calories=40
\end{lstlisting}

The  optimizer  will  not  consider  the  index  on  age  as  it  is  misled  by  the  OR predicate. To make it consider the index we can use UNION instead or an OR.

\begin{lstlisting}
SELECT type	FROM CHEESE	WHERE calories=90
UNION
SELECT type	FROM CHEESE	WHERE calories=40
\end{lstlisting}

{\color{gray}\subsection{} No index is available:}

\begin{lstlisting}
SELECT DISTINCT CheeseId, type
FROM CHEESE 
\end{lstlisting}

Tuples containing the primary key will be by default unique, so the distinct operation is needless. 

\begin{lstlisting}
SELECT CheeseId, type
FROM CHEESE 
\end{lstlisting}

{\color{gray}\subsection{} No index is available:}

\begin{lstlisting}
SELECT AVG (proteins)
FROM CHEESE
GROUP BY producer
HAVING type = "Alverca"
\end{lstlisting}

By grouping cheeses by producer and filtering by type only after, most of groups with type "Alverca" will be discarded, to improve the performance of this query, the filter condition by type should be first in order to the group operation process less tuples, in this case the filter condition is applied to tuples instead of groups so the HAVING condition is swapped by WHERE condition.

\begin{lstlisting}
SELECT AVG (proteins)
FROM CHEESE
WHERE type = "Alverca"
GROUP BY producer
\end{lstlisting}


