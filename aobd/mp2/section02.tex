\subsection{}

{	\color{gray}
Consider the following schedule for three concurrent transactions:
}
{
	\color{gray}
	\begin{table}[H]
\centering
{\color{gray}
	\begin{tabular}{l|l|l|l|}
	\cline{2-4}
		                    & T1       & T2       & T3       \\ \hline
	\multicolumn{1}{|l|}{1} & write(A) &          &          \\ \hline
	\multicolumn{1}{|l|}{2} &          &          & write(B) \\ \hline
	\multicolumn{1}{|l|}{3} &          & write(A) &          \\ \hline
	\multicolumn{1}{|l|}{4} & write(B) &          &          \\ \hline
	\multicolumn{1}{|l|}{5} &          & write(B) &          \\ \hline
	\end{tabular}
}
	\end{table}

}

\subsubsection{}
{\color{gray}Is the schedule allowed in Strict 2-Phase Locking? Justify.}
\begin{center}
\begin{tikzpicture}[->,>=stealth',shorten >=1pt,auto,node distance=3cm,
  thick,main node/.style={circle,draw,font=\sffamily\Large\bfseries}]

  \node[main node] (1) {$T_1$};
  \node[main node] (2) [right of=1] {$T_2$};
  \node[main node] (3) [right of=2] {$T_3$};


  \path[every node/.style={font=\sffamily\small}]
    (2) edge [bend left] node[below] {3} (1)
    (2) edge [bend right] node[below] {5} (3)
    (1) edge [bend left] node {4} (3);
\end{tikzpicture}
\end{center}
	
	Yes, strict 2-Phase Locking allows only schedules whose precedence graph is acyclic, in this case the graph is acyclic.

\subsubsection{}

{\color{gray}Is the schedule allowed by the timestamp-based protocol? Justify.}

TS($T_1$)=1

TS($T_2$)=3

TS($T_3$)=2

\begin{table}[h]
\centering
\begin{tabular}{c|c|c|c|c|}
\cline{2-5}
                        & \multicolumn{2}{c|}{A} & \multicolumn{2}{c|}{B} \\ \hline
\multicolumn{1}{|c|}{t} & Rts(A)    & Wts(A)   & Rts(B)    & Wts(B)   \\ \hline
\multicolumn{1}{|c|}{1} & 0          & 1         & 0          & 0         \\ \hline
\multicolumn{1}{|c|}{2} & 0          & 1         & 0          & 2         \\ \hline
\multicolumn{1}{|c|}{3} & 0          & 3         & 0          & 2         \\ \hline
\multicolumn{1}{|c|}{4} &            &           &            &           \\ \hline
\multicolumn{1}{|c|}{5} &            &           &            &           \\ \hline
\end{tabular}
\end{table}

No, at $t=4$ transaction $T_1$ issues \textbf{write}(B), $[ TS(T_1)=1 ] < [ Wts(B) = 2 ]$, which means that $T_1$ is attempting to write an obsolete value of B, hence this write operation is rejected, $T_1$ is rolled back. 


\subsection{}
{\color{gray}Consider the following schedule for two concurrent transactions:}

\begin{table}[h]
\centering
{\color{gray}
\begin{tabular}{|c|c|c|}
\hline
  & T1        & T2        \\ \hline
1 & lock-S(A) &           \\ \hline
2 & read(A)   &           \\ \hline
3 & unlock(A) &           \\ \hline
4 &           & lock-X(B) \\ \hline
5 &           & write(B)  \\ \hline
6 &           & unlock(B) \\ \hline
7 & lock-S(B) &           \\ \hline
8 & read(B)   &           \\ \hline
9 & unlock(B) &           \\ \hline
\end{tabular}
}
\end{table}

\subsubsection{}
{\color{gray}Is the schedule allowed in Strict 2-Phase Locking? Justify.}

{\color{red}XXXXXXXXXXXXXXXXXXXXXXXXXXXXXXXXXXXXXXXXX}
\subsubsection{}
{\color{gray}Is the schedule allowed by the timestamp-based protocol? Justify.}

{\color{red}XXXXXXXXXXXXXXXXXXXXXXXXXXXXXXXXXXXXXXXXX}
