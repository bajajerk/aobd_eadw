{\color{gray} Consider the following two queries:}
\begin{lstlisting}[frame=single]
Q1: 	SELECT type, producer
	FROM CHEESE NATURAL JOIN PRODUCTION NATURAL JOIN PROVENANCE NATURAL JOIN REGION
	WHERE country = 'Portugal' AND amount > 10,000 AND season = 'Winter'
\end{lstlisting}

\begin{lstlisting}[frame=single]
Q2: 	SELECT type, COUNT(*)
	FROM CHEESE
	GROUP BY type;
\end{lstlisting}

\subsection{}
{\color{gray}Which of these queries corresponds to a typical access pattern of OLTP applications? And which one corresponds to a typical access pattern of OLAP queries? Justify your answer.}

	OLAP applications typically involves agreggations and complex queries which is the case of Query 1. 

	OLTP applications typically involves simple queries that returns relatively few records, which is the case of Query 2.

\subsection{}
{\color{gray}In which of the queries would it make sense to use a low isolation level (e.g. READ UNCOMMITED)? And a higher isolation level (e.g. SERIALIZABLE)? Justify your answer.}

The first query does natural joins between four tables which is an indicator that this query should use an high isolation level such as SERIALIZABLE.

On the other hand, the second query does a simple group by "Type" and changes to this table would produce just different counts, this type of query could require a lower isolation level.

\subsection{}
{\color{gray} Consider the different tuning levels considered in the lectures (schema, index, queries, etc). For the first query, indicate two possible optimizations at two different levels.}

Queries formadas a partir de varios natural joins, criam tabelas com um tamanho enorme o que torna as pesquisas mais caras e lentas. Uma forma de otimizar este problema é criar diferentes tabelas temporários com os dados necessários para responder às condições desejadas, por exemplo, a criação de uma query temporária sobre a tabela que responda à condição "WHERE country = 'Portugal'", outra em relação à tabela que responde à condição "WHERE amount > 1000" e por ultimo uma sobre a tabela que responde à condição "WHERE season = 'Winter'".

Após esta otimização é ainda possível criar um indice sobre os atributos desejados, "country", "amount" e "season" de forma a acelerar as pesquisas. 
