1 - Alcobaça

2 - Alverca

3 - Azeitão

4 - Camembert

5 - Creme

6 - Emmental

7 - Évora

8 - Flamengo

9 - Fresco

10 - Ilha

11 - Parmesão

12 - Quark

13 - Requeijão

14 - Roquefort

15 - Serpa

16 - Serra

	{\color{gray}Consider the problem of inserting the following keys, in the given order, into an empty B+-tree where nodes can hold up to 3 values:}
	
	{\color{gray}Parmesão, Ilha, Camembert, Fresco, Requeijão, Azeitão, Alverca, Serra, Alcobaça, Roquefort, Flamengo, Emmental, Évora, Creme, Serpa, Quark}

	\subsection{}
	{\color{gray}Draw the tree after each insertion.}
 		Insertion order: 1, 10, 4, 9, 13, 3, 2, 16, 1, 14, 8, 6, 7, 5, 15, 12

\begin{center}
\begin{tikzpicture}
\tikzstyle{bplus}=[rectangle split, rectangle split horizontal,rectangle split ignore empty parts,draw]
\tikzstyle{every node}=[bplus]
\tikzstyle{level 1}=[sibling distance=15mm]
\tikzstyle{level 2}=[sibling distance=15mm]
\node[fill=lightgray] {11} [->]
;\end{tikzpicture}
\end{center}


\begin{center}
\begin{tikzpicture}
\tikzstyle{bplus}=[rectangle split, rectangle split horizontal,rectangle split ignore empty parts,draw]
\tikzstyle{every node}=[bplus]
\tikzstyle{level 1}=[sibling distance=15mm]
\tikzstyle{level 2}=[sibling distance=15mm]
\node {10\nodepart{two} 11} [->]
;\end{tikzpicture}
\end{center}


\begin{center}
\begin{tikzpicture}
\tikzstyle{bplus}=[rectangle split, rectangle split horizontal,rectangle split ignore empty parts,draw]
\tikzstyle{every node}=[bplus]
\tikzstyle{level 1}=[sibling distance=15mm]
\tikzstyle{level 2}=[sibling distance=15mm]
\node {10} [-]
  child {node (a) {4}}
  child {node (b) {10 \nodepart{two} 11}} 
;
\draw[->] (a) -- (b);
\end{tikzpicture}
\end{center}

\begin{center}
\begin{tikzpicture}
\tikzstyle{bplus}=[rectangle split, rectangle split horizontal,rectangle split ignore empty parts,draw]
\tikzstyle{every node}=[bplus]
\tikzstyle{level 1}=[sibling distance=15mm]
\tikzstyle{level 2}=[sibling distance=15mm]
\node {10} [-]
  child {node (a) {4 \nodepart{two} 9}}
  child {node (b) {10 \nodepart{two} 11}} 
;
\draw[->] (a) -- (b);
\end{tikzpicture}
\end{center}

\begin{center}
\begin{tikzpicture}
\tikzstyle{bplus}=[rectangle split, rectangle split horizontal,rectangle split ignore empty parts,draw]
\tikzstyle{every node}=[bplus]
\tikzstyle{level 1}=[sibling distance=15mm]
\tikzstyle{level 2}=[sibling distance=15mm]
\node {10 \nodepart{two} 11} [-]
  child {node (a) {4 \nodepart{two} 9}}
  child {node (b) {10}}
  child {node (c) {11 \nodepart{two} 13}} 
;
\draw[->] (a) -- (b);
\draw[->] (b) -- (c);
\end{tikzpicture}
\end{center}


	\subsection{}
	{\color{gray}Delete the following keys from the B+tree data structure from the previous exercise: Ilha; Flamengo; Emmental; Serpa. Draw the tree after each deletion.}

 		Deletion order: 10, 8, 6, 15



\begin{center}
\begin{tikzpicture}
\tikzstyle{bplus}=[rectangle split, rectangle split horizontal,rectangle split ignore empty parts,draw]
\tikzstyle{every node}=[bplus]
\tikzstyle{level 1}=[sibling distance=60mm]
\tikzstyle{level 2}=[sibling distance=15mm]
\node {11} [->]
  child {node {3 \nodepart{two} 7}
    child {node {1 \nodepart{two} 2}}
    child {node {4 \nodepart{two} 6}}
    child {node {8 \nodepart{two} 9}}    
  } 
  child {node {21 \nodepart{two} 28 \nodepart{three} 32 \nodepart{four} 50}
    child {node {17 \nodepart{two} 20}}
    child {node {22 \nodepart{two} 25}}
    child {node {28 \nodepart{two} 30}}    
    child[sibling distance=25mm] {node {34 \nodepart{two} 38 \nodepart{three} 44 \nodepart{four} 47}}    
    child[sibling distance=25mm] {node {53 \nodepart{two} 54 \nodepart{three} 60 \nodepart{four} 88}}    
  }
;\end{tikzpicture}
\end{center}
