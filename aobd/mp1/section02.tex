

	{\color{gray}Consider the problem of inserting the following keys, in the given order, into an empty B+-tree where nodes can hold up to 3 values:}
	
	{\color{gray}Parmesão, Ilha, Camembert, Fresco, Requeijão, Azeitão, Alverca, Serra, Alcobaça, Roquefort, Flamengo, Emmental, Évora, Creme, Serpa, Quark}

1 - Alcobaça

2 - Alverca

3 - Azeitão

4 - Camembert

5 - Creme

6 - Emmental

7 - Évora

8 - Flamengo

9 - Fresco

10 - Ilha

11 - Parmesão

12 - Quark

13 - Requeijão

14 - Roquefort

15 - Serpa

16 - Serra

	\subsection{}
	{\color{gray}Draw the tree after each insertion.}
 		Insertion order: 1, 10, 4, 9, 13, 3, 2, 16, 1, 14, 8, 6, 7, 5, 15, 12



\textbf{Insert 11 (Parmesão)}
\begin{center}
\begin{tikzpicture}
\tikzstyle{bplus}=[rectangle split, rectangle split horizontal,rectangle split ignore empty parts,draw]
\tikzstyle{every node}=[bplus]
\tikzstyle{level 1}=[sibling distance=15mm]
\tikzstyle{level 2}=[sibling distance=15mm]
\node[fill=lightgray] {11} [->]
;\end{tikzpicture}
\end{center}

\textbf{Insert 10 (Ilha)}
\begin{center}
\begin{tikzpicture}
\tikzstyle{bplus}=[rectangle split, rectangle split horizontal,rectangle split ignore empty parts,draw]
\tikzstyle{every node}=[bplus]
\tikzstyle{level 1}=[sibling distance=15mm]
\tikzstyle{level 2}=[sibling distance=15mm]
\node {10\nodepart{two} 11} [->]
;\end{tikzpicture}
\end{center}

\textbf{Insert 4 (Camembert)}
\begin{center}
\begin{tikzpicture}
\tikzstyle{bplus}=[rectangle split, rectangle split horizontal,rectangle split ignore empty parts,draw]
\tikzstyle{every node}=[bplus]
\tikzstyle{level 1}=[sibling distance=15mm]
\tikzstyle{level 2}=[sibling distance=15mm]
\node {10} [-]
  child {node (a) {4}}
  child {node (b) {10 \nodepart{two} 11}} 
;
\draw[->] (a) -- (b);
\end{tikzpicture}
\end{center}

\textbf{Insert 9 (Fresco)}
\begin{center}
\begin{tikzpicture}
\tikzstyle{bplus}=[rectangle split, rectangle split horizontal,rectangle split ignore empty parts,draw]
\tikzstyle{every node}=[bplus]
\tikzstyle{level 1}=[sibling distance=15mm]
\tikzstyle{level 2}=[sibling distance=15mm]
\node {10} [-]
  child {node (a) {4 \nodepart{two} 9}}
  child {node (b) {10 \nodepart{two} 11}} 
;
\draw[->] (a) -- (b);
\end{tikzpicture}
\end{center}

\textbf{Insert 13 (Requeijão)}
\begin{center}
\begin{tikzpicture}
\tikzstyle{bplus}=[rectangle split, rectangle split horizontal,rectangle split ignore empty parts,draw]
\tikzstyle{every node}=[bplus]
\tikzstyle{level 1}=[sibling distance=15mm]
\tikzstyle{level 2}=[sibling distance=15mm]
\node {10 \nodepart{two} 11} [-]
  child {node (a) {4 \nodepart{two} 9}}
  child {node (b) {10}}
  child {node (c) {11 \nodepart{two} 13}} 
;
\draw[->] (a) -- (b);
\draw[->] (b) -- (c);
\end{tikzpicture}
\end{center}

\textbf{Insert 3 (Azeitão)}
\begin{center}
\begin{tikzpicture}
\tikzstyle{bplus}=[rectangle split, rectangle split horizontal,rectangle split ignore empty parts,draw]
\tikzstyle{every node}=[bplus]
\tikzstyle{level 1}=[sibling distance=30mm]
\tikzstyle{level 2}=[sibling distance=15mm]
\node {10} [-]
  child {node (a) {4}
    child {node (b) {3}}
    child {node (c) {4 \nodepart{two} 9}}
  } 
  child {node (d) {11}
    child {node (e) {10}}
    child {node (f) {11 \nodepart{two} 13}}
  }
;
\draw[->] (b) -- (c);
\draw[->] (c) -- (e);
\draw[->] (e) -- (f);
\end{tikzpicture}
\end{center}

\textbf{Insert 2 (Alverca)}
\begin{center}
\begin{tikzpicture}
\tikzstyle{bplus}=[rectangle split, rectangle split horizontal,rectangle split ignore empty parts,draw]
\tikzstyle{every node}=[bplus]
\tikzstyle{level 1}=[sibling distance=30mm]
\tikzstyle{level 2}=[sibling distance=15mm]
\node {10} [-]
  child {node (a) {4}
    child {node (b) {2 \nodepart{two} 3}}
    child {node (c) {4 \nodepart{two} 9}}
  } 
  child {node (d) {11}
    child {node (e) {10}}
    child {node (f) {11 \nodepart{two} 13}}
  }
;
\draw[->] (b) -- (c);
\draw[->] (c) -- (e);
\draw[->] (e) -- (f);
\end{tikzpicture}
\end{center}

\textbf{Insert 16 (Serra)}
\begin{center}
\begin{tikzpicture}
\tikzstyle{bplus}=[rectangle split, rectangle split horizontal,rectangle split ignore empty parts,draw]
\tikzstyle{every node}=[bplus]
\tikzstyle{level 1}=[sibling distance=40mm]
\tikzstyle{level 2}=[sibling distance=15mm]
\node {10} [-]
  child {node (a) {4}
    child {node (b) {2 \nodepart{two} 3}}
    child {node (c) {4 \nodepart{two} 9}}
  } 
  child {node (d) {11 \nodepart{two} 13}
    child {node (e) {10}}
    child {node (f) {11}}
    child {node (g) {13 \nodepart{two} 16}}
  }
;
\draw[->] (b) -- (c);
\draw[->] (c) -- (e);
\draw[->] (e) -- (f);
\draw[->] (f) -- (g);
\end{tikzpicture}
\end{center}

\textbf{Insert 1 (Alcobaça)}
\begin{center}
\begin{tikzpicture}
\tikzstyle{bplus}=[rectangle split, rectangle split horizontal,rectangle split ignore empty parts,draw]
\tikzstyle{every node}=[bplus]
\tikzstyle{level 1}=[sibling distance=40mm]
\tikzstyle{level 2}=[sibling distance=15mm]
\node {10} [-]
  child {node {2 \nodepart{two} 4}
    child {node (a) {1}}
    child {node (b) {2 \nodepart{two} 3}}
    child {node (c) {4 \nodepart{two} 9}}
  } 
  child {node {11 \nodepart{two} 13}
    child {node (d) {10}}
    child {node (e) {11}}
    child {node (f) {13 \nodepart{two} 16}}
  }
;
\draw[->] (a) -- (b);
\draw[->] (b) -- (c);
\draw[->] (c) -- (d);
\draw[->] (d) -- (e);
\draw[->] (e) -- (f);
\end{tikzpicture}
\end{center}

\textbf{Insert 14 (Roquefort)}
\begin{center}
\begin{tikzpicture}
\tikzstyle{bplus}=[rectangle split, rectangle split horizontal,rectangle split ignore empty parts,draw]
\tikzstyle{every node}=[bplus]
\tikzstyle{level 1}=[sibling distance=40mm]
\tikzstyle{level 2}=[sibling distance=15mm]
\node {10 \nodepart{two} 13} [-]
  child {node (a) {2 \nodepart{two} 4}
    child {node (b) {1}}
    child {node (c) {2 \nodepart{two} 3}}
    child {node (d) {4 \nodepart{two} 9}}
  } 
  child {node (e) {11}
    child {node (f) {10}}
    child {node (g) {11}}
  }
  child {node (h) {14}
    child {node (i) {13}}
    child {node (j) {14 \nodepart{two} 16}}
  }
;
\draw[->] (b) -- (c);
\draw[->] (c) -- (d);
\draw[->] (d) -- (f);
\draw[->] (f) -- (g);
\draw[->] (g) -- (i);
\draw[->] (i) -- (j);
\end{tikzpicture}
\end{center}

\textbf{Insert 8 (Flamengo)}
\begin{center}
\begin{tikzpicture}
\tikzstyle{bplus}=[rectangle split, rectangle split horizontal,rectangle split ignore empty parts,draw]
\tikzstyle{every node}=[bplus]
\tikzstyle{level 1}=[sibling distance=60mm]
\tikzstyle{level 2}=[sibling distance=30mm]
\tikzstyle{level 3}=[sibling distance=15mm]
\node {10} [-]
  child {node {4}
    child {node {2}
       child {node (a) {1}}
       child {node (b) {2 \nodepart{two} 3}}
    }
    child {node {8}
        child {node (c) {4}}
        child {node (d) {8 \nodepart{two} 9}}
    }
  } 
  child {node {13}
    child {node {11}
       child {node (e) {10}}
       child {node (f) {11}}
    }
    child {node {14}
        child {node (g) {13}}
        child {node (h) {14 \nodepart{two} 16}}
    }
  }
;
\draw[->] (a) -- (b);
\draw[->] (b) -- (c);
\draw[->] (c) -- (d);
\draw[->] (d) -- (e);
\draw[->] (e) -- (f);
\draw[->] (f) -- (g);
\draw[->] (g) -- (h);
\end{tikzpicture}
\end{center}

\textbf{Insert 6 (Emmental)}
\begin{center}
\begin{tikzpicture}
\tikzstyle{bplus}=[rectangle split, rectangle split horizontal,rectangle split ignore empty parts,draw]
\tikzstyle{every node}=[bplus]
\tikzstyle{level 1}=[sibling distance=60mm]
\tikzstyle{level 2}=[sibling distance=30mm]
\tikzstyle{level 3}=[sibling distance=15mm]
\node {10} [-]
  child {node {4}
    child {node {2}
       child {node (a) {1}}
       child {node (b) {2 \nodepart{two} 3}}
    }
    child {node {8}
        child {node (c) {4 \nodepart{two} 6}}
        child {node (d) {8 \nodepart{two} 9}}
    }
  } 
  child {node {13}
    child {node {11}
       child {node (e) {10}}
       child {node (f) {11}}
    }
    child {node {14}
        child {node (g) {13}}
        child {node (h) {14 \nodepart{two} 16}}
    }
  }
;
\draw[->] (a) -- (b);
\draw[->] (b) -- (c);
\draw[->] (c) -- (d);
\draw[->] (d) -- (e);
\draw[->] (e) -- (f);
\draw[->] (f) -- (g);
\draw[->] (g) -- (h);
\end{tikzpicture}
\end{center}

\newpage
\textbf{Insert 7 (Évora)}
\begin{center}
\begin{tikzpicture}
\tikzstyle{bplus}=[rectangle split, rectangle split horizontal,rectangle split ignore empty parts,draw]
\tikzstyle{every node}=[bplus]
\tikzstyle{level 1}=[sibling distance=75mm]
\tikzstyle{level 2}=[sibling distance=40mm]
\tikzstyle{level 3}=[sibling distance=15mm]
\node {10} [-]
  child {node {4}
    child {node {2}
       child {node (a) {1}}
       child {node (b) {2 \nodepart{two} 3}}
    }
    child {node {6 \nodepart{two} 8}
        child {node (c) {4}}
        child {node (d) {6 \nodepart{two} 7}}
        child {node (e) {8 \nodepart{two} 9}}
    }
  } 
  child {node {13}
    child {node {11}
       child {node (f) {10}}
       child {node (g) {11}}
    }
    child {node {14}
        child {node (h) {13}}
        child {node (i) {14 \nodepart{two} 16}}
    }
  }
;
\draw[->] (a) -- (b);
\draw[->] (b) -- (c);
\draw[->] (c) -- (d);
\draw[->] (d) -- (e);
\draw[->] (e) -- (f);
\draw[->] (f) -- (g);
\draw[->] (g) -- (h);
\draw[->] (h) -- (i);
\end{tikzpicture}
\end{center}

\textbf{Insert 5 (Creme)}
\begin{center}
\begin{tikzpicture}
\tikzstyle{bplus}=[rectangle split, rectangle split horizontal,rectangle split ignore empty parts,draw]
\tikzstyle{every node}=[bplus]
\tikzstyle{level 1}=[sibling distance=75mm]
\tikzstyle{level 2}=[sibling distance=40mm]
\tikzstyle{level 3}=[sibling distance=15mm]
\node {10} [-]
  child {node {4}
    child {node {2}
       child {node (a) {1}}
       child {node (b) {2 \nodepart{two} 3}}
    }
    child {node {6 \nodepart{two} 8}
        child {node (c) {4 \nodepart{two} 5}}
        child {node (d) {6 \nodepart{two} 7}}
        child {node (e) {8 \nodepart{two} 9}}
    }
  } 
  child {node {13}
    child {node {11}
       child {node (f) {10}}
       child {node (g) {11}}
    }
    child {node {14}
        child {node (h) {13}}
        child {node (i) {14 \nodepart{two} 16}}
    }
  }
;
\draw[->] (a) -- (b);
\draw[->] (b) -- (c);
\draw[->] (c) -- (d);
\draw[->] (d) -- (e);
\draw[->] (e) -- (f);
\draw[->] (f) -- (g);
\draw[->] (g) -- (h);
\draw[->] (h) -- (i);
\end{tikzpicture}
\end{center}

\newpage
\textbf{Insert 15 (Serpa)}
\begin{center}
\begin{tikzpicture}
\tikzstyle{bplus}=[rectangle split, rectangle split horizontal,rectangle split ignore empty parts,draw]
\tikzstyle{every node}=[bplus]
\tikzstyle{level 1}=[sibling distance=80mm]
\tikzstyle{level 2}=[sibling distance=45mm]
\tikzstyle{level 3}=[sibling distance=15mm]
\node {10} [-]
  child {node {4}
    child {node {2}
       child {node (a) {1}}
       child {node (b) {2 \nodepart{two} 3}}
    }
    child {node {6 \nodepart{two} 8}
        child {node (c) {4 \nodepart{two} 5}}
        child {node (d) {6 \nodepart{two} 7}}
        child {node (e) {8 \nodepart{two} 9}}
    }
  } 
  child {node {13}
    child {node {11}
       child {node (f) {10}}
       child {node (g) {11}}
    }
    child {node {14 \nodepart{two} 15}
        child {node (h) {13}}
        child {node (i) {14}}
        child {node (j) {15 \nodepart{two} 16}}
    }
  }
;
\draw[->] (a) -- (b);
\draw[->] (b) -- (c);
\draw[->] (c) -- (d);
\draw[->] (d) -- (e);
\draw[->] (e) -- (f);
\draw[->] (f) -- (g);
\draw[->] (g) -- (h);
\draw[->] (h) -- (i);
\draw[->] (i) -- (j);

\end{tikzpicture}
\end{center}

\textbf{Insert 12 (Quark)}
\begin{center}
\begin{tikzpicture}
\tikzstyle{bplus}=[rectangle split, rectangle split horizontal,rectangle split ignore empty parts,draw]
\tikzstyle{every node}=[bplus]
\tikzstyle{level 1}=[sibling distance=80mm]
\tikzstyle{level 2}=[sibling distance=45mm]
\tikzstyle{level 3}=[sibling distance=15mm]
\node {10} [-]
  child {node {4}
    child {node {2}
       child {node (a) {1}}
       child {node (b) {2 \nodepart{two} 3}}
    }
    child {node {6 \nodepart{two} 8}
        child {node (c) {4 \nodepart{two} 5}}
        child {node (d) {6 \nodepart{two} 7}}
        child {node (e) {8 \nodepart{two} 9}}
    }
  } 
  child {node {13}
    child {node {11}
       child {node (f) {10}}
       child {node (g) {11 \nodepart{two} 12}}
    }
    child {node {14 \nodepart{two} 15}
        child {node (h) {13}}
        child {node (i) {14}}
        child {node (j) {15 \nodepart{two} 16}}
    }
  }
;
\draw[->] (a) -- (b);
\draw[->] (b) -- (c);
\draw[->] (c) -- (d);
\draw[->] (d) -- (e);
\draw[->] (e) -- (f);
\draw[->] (f) -- (g);
\draw[->] (g) -- (h);
\draw[->] (h) -- (i);
\draw[->] (i) -- (j);
\end{tikzpicture}
\end{center}

	\subsection{}
	{\color{gray}Delete the following keys from the B+tree data structure from the previous exercise: Ilha; Flamengo; Emmental; Serpa. Draw the tree after each deletion.}

 		Deletion order: 10, 8, 6, 15

\textbf{Delete 10 (Ilha)}
\begin{center}
\begin{tikzpicture}
\tikzstyle{bplus}=[rectangle split, rectangle split horizontal,rectangle split ignore empty parts,draw]
\tikzstyle{every node}=[bplus]
\tikzstyle{level 1}=[sibling distance=80mm]
\tikzstyle{level 2}=[sibling distance=45mm]
\tikzstyle{level 3}=[sibling distance=15mm]
\node {11} [-]
  child {node {4}
    child {node {2}
       child {node (a) {1}}
       child {node (b) {2 \nodepart{two} 3}}
    }
    child {node {6 \nodepart{two} 8}
        child {node (c) {4 \nodepart{two} 5}}
        child {node (d) {6 \nodepart{two} 7}}
        child {node (e) {8 \nodepart{two} 9}}
    }
  } 
  child {node {13}
    child {node {11}
       child {node (f) {11}}
       child {node (g) {12}}
    }
    child {node {14 \nodepart{two} 16}
        child {node (h) {13}}
        child {node (i) {14}}
        child {node (j) {15 \nodepart{two} 16}}
    }
  }
;
\draw[->] (a) -- (b);
\draw[->] (b) -- (c);
\draw[->] (c) -- (d);
\draw[->] (d) -- (e);
\draw[->] (e) -- (f);
\draw[->] (f) -- (g);
\draw[->] (g) -- (h);
\draw[->] (h) -- (i);
\draw[->] (i) -- (j);
\end{tikzpicture}
\end{center}

\newpage
\textbf{Delete 8 (Flamengo)}
\begin{center}
\begin{tikzpicture}
\tikzstyle{bplus}=[rectangle split, rectangle split horizontal,rectangle split ignore empty parts,draw]
\tikzstyle{every node}=[bplus]
\tikzstyle{level 1}=[sibling distance=80mm]
\tikzstyle{level 2}=[sibling distance=45mm]
\tikzstyle{level 3}=[sibling distance=15mm]
\node {11} [-]
  child {node {4}
    child {node {2}
       child {node (a) {1}}
       child {node (b) {2 \nodepart{two} 3}}
    }
    child {node {6 \nodepart{two} 9}
        child {node (c) {4 \nodepart{two} 5}}
        child {node (d) {6 \nodepart{two} 7}}
        child {node (e) {9}}
    }
  } 
  child {node {13}
    child {node {11}
       child {node (f) {11}}
       child {node (g) {12}}
    }
    child {node {14 \nodepart{two} 16}
        child {node (h) {13}}
        child {node (i) {14}}
        child {node (j) {15 \nodepart{two} 16}}
    }
  }
;
\draw[->] (a) -- (b);
\draw[->] (b) -- (c);
\draw[->] (c) -- (d);
\draw[->] (d) -- (e);
\draw[->] (e) -- (f);
\draw[->] (f) -- (g);
\draw[->] (g) -- (h);
\draw[->] (h) -- (i);
\draw[->] (i) -- (j);
\end{tikzpicture}
\end{center}

\textbf{Delete 6 (Emmental)}
\begin{center}
\begin{tikzpicture}
\tikzstyle{bplus}=[rectangle split, rectangle split horizontal,rectangle split ignore empty parts,draw]
\tikzstyle{every node}=[bplus]
\tikzstyle{level 1}=[sibling distance=80mm]
\tikzstyle{level 2}=[sibling distance=45mm]
\tikzstyle{level 3}=[sibling distance=15mm]
\node {11} [-]
  child {node {4}
    child {node {2}
       child {node (a) {1}}
       child {node (b) {2 \nodepart{two} 3}}
    }
    child {node {7 \nodepart{two} 9}
        child {node (c) {4 \nodepart{two} 5}}
        child {node (d) {7}}
        child {node (e) {9}}
    }
  } 
  child {node {13}
    child {node {11}
       child {node (f) {11}}
       child {node (g) {12}}
    }
    child {node {14 \nodepart{two} 16}
        child {node (h) {13}}
        child {node (i) {14}}
        child {node (j) {15 \nodepart{two} 16}}
    }
  }
;
\draw[->] (a) -- (b);
\draw[->] (b) -- (c);
\draw[->] (c) -- (d);
\draw[->] (d) -- (e);
\draw[->] (e) -- (f);
\draw[->] (f) -- (g);
\draw[->] (g) -- (h);
\draw[->] (h) -- (i);
\draw[->] (i) -- (j);
\end{tikzpicture}
\end{center}

\newpage
\textbf{Delete 15 (Serpa)}
\begin{center}
\begin{tikzpicture}
\tikzstyle{bplus}=[rectangle split, rectangle split horizontal,rectangle split ignore empty parts,draw]
\tikzstyle{every node}=[bplus]
\tikzstyle{level 1}=[sibling distance=80mm]
\tikzstyle{level 2}=[sibling distance=45mm]
\tikzstyle{level 3}=[sibling distance=15mm]
\node {11} [-]
  child {node {4}
    child {node {2}
       child {node (a) {1}}
       child {node (b) {2 \nodepart{two} 3}}
    }
    child {node {7 \nodepart{two} 9}
        child {node (c) {4 \nodepart{two} 5}}
        child {node (d) {7}}
        child {node (e) {9}}
    }
  } 
  child {node {13}
    child {node {11}
       child {node (f) {11}}
       child {node (g) {12}}
    }
    child {node {14 \nodepart{two} 16}
        child {node (h) {13}}
        child {node (i) {14}}
        child {node (j) {16}}
    }
  }
;
\draw[->] (a) -- (b);
\draw[->] (b) -- (c);
\draw[->] (c) -- (d);
\draw[->] (d) -- (e);
\draw[->] (e) -- (f);
\draw[->] (f) -- (g);
\draw[->] (g) -- (h);
\draw[->] (h) -- (i);
\draw[->] (i) -- (j);
\end{tikzpicture}
\end{center}


